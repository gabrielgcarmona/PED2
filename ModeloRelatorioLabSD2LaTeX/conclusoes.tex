\section{Discuss�o e Conclus�es}
Jamais esque�a este item! Neste item, descreva resumidamente os resultados observados e os seus significados. Exemplo: Com o aumento da frequ�ncia, observou-se que a tens�o de sa�da foi caindo. Isto aconteceu porque trata-se de um filtro passa baixas, o qual apresenta esta caracter�stica. Neste item os integrantes do grupo discutem o porqu� dos resultados obtidos, buscando demonstrar que eles atendem ao que foi solicitado e comprovam o sucesso do experimento.
Compara��es com valores obtidos por outros, em artigos, manuais ou \emph{data-sheets}, bem como sua compara��o com o que � esperado teoricamente, ajudam a comprovar o sucesso do experimento \cite{Henker1999}, \cite{Voit2010}.