\section{Resultados}
Os resultados devem ser apresentados numa sequ�ncia que os correlacione com o experimento descrito na se��o anterior. Neste item os integrantes do grupo mostram os resultados em forma de tabela, gr�ficos, ou de acordo com a necessidade. Aqui tamb�m deve ser feita uma an�lise sobre cada um desses resultados. A forma das curvas, o valor lido nos instrumentos, etc. Nunca deixe um gr�fico ou uma tabela sem a devida interpreta��o! Um erro comum � colocar 2 ou mais gr�ficos e n�o especificar os porqu�s do que foi medido. Caso voc� perceba que algo aconteceu em laborat�rio que n�o est� de acordo com a teoria procure avaliar as raz�es.

